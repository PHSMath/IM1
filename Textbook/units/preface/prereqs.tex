\section{Prerequisites}
In order to be successful in our Integrated Math I course, there are prerequisite skills that you should have already mastered prior to starting this content. Prerequisites from prior units in this class are not listed. It is assumed that you will not proceed to the next unit until you have complete the content of the previous unit(s). Instead of listing these skills by year obtained, they are provided by the unit in which the skills are needed. If you need assistance with these skills, please see the Pacific Middle/High School Math Review book.
\subsection{General Prerequisites}
\begin{enumerate}
    \item Multiplication Facts 1 through 13
    \item Division Facts
    \item Prime Factorization, Greatest Common Denominator, Least Common Multiple
    \item Perfect Squares (1-13, 15, 16, 20, 25)
    \item Perfect Cubes (1-6, 10)
    \item Pythagorean Theorem and Distance Formula
    \item Fractions - Add, Subtract, Multiply, and Divide
    \item Fractions - Convert from and to mixed numbers/improper fractions
    \item Properties of Real Numbers
\end{enumerate}
\subsection{Unit One - Linear Equations and Inequalities}
\begin{enumerate}
    \item Order of Operations
    \item Evaluating Expressions
    \item Solving Equations
    \item Rates of Change, Proportional Relationships
    \item Graphing points on Coordinate Plane
\end{enumerate}
\subsection{Unit Two - Exponential Functions}
\begin{enumerate}
    \item Exponent Rules
    \item Scientific Notation
\end{enumerate}
\subsection{Unit Three - Data Analysis and Reasoning}
\begin{enumerate}
    \item Scatter Plots
    \item Line of Best Fit
    \item Measures of Center - Mean, Median, Mode
    \item Measures of Spread - Deviation, Absolute Deviation, Mean Absolute Deviation, Quartiles, Range, Interquartile Range
    \item Two-way Tables
\end{enumerate}
\subsection{Unit Four - Basics of Geometry}
\begin{enumerate}
    \item Transformations - Rotation, Reflection, Translation, Dilation
    \item Classifying Angles and Triangles
    \item Angle Pairs
    \item Area, Surface Area, and Volume for Geometric Shapes
\end{enumerate}