\chapter{Linear Equations and Inequalities}\label{chapter:linear}
\section{Linear Equations - Overview and Definitions}
In math, a \hl{\textbf{term}} is a constant, a variable (or variable raised to a power), or a constant times a variable. Some examples of terms are $3x, -5, y, $ and $22$. The letter portion of a term is called a \hl{\textbf{variable}} and the number portion is called the variable's \hl{\textbf{coefficient}}. For the term $-2x$, $x$ is the variable and $-2$ is the coefficient. An \hl{\textbf{expression}} is one or more terms that are added together (we will be considering subtraction as adding a term that has a negative coefficient). Examples of expressions are: $-4x$ (yes, a single term is an expression), $5x^2-3x+4$, and $2$. 
 
An \hl{\textbf{equation}} is two expressions combined with an equal sign which form a statement that both expressions are equal to each other. We will primarily be focusing on linear equations. A \hl{\textbf{linear equation in one variable}} is an equation that can be written in the form \sethlcolor{pink}\hl{$ax+b =0$} where $a$ and $b$ are constants and $a\neq 0$. We solved equations of this type in 8th Grade Math. This year we will extend solving equations to \sethlcolor{yellow}\hl{\textbf{linear equations of in two variables}}. These are equations that can be written in the form \sethlcolor{pink}\hl{$ax+by+c=0$}\sethlcolor{yellow} where $a, b,$ and $c$ are constants and both $a$ and $b$ may not be zero at the same time.
