\section{Solving Two-Step Equations}
Two-step equations are those that are in the form \sethlcolor{pink}\hl{$Ax\pm B = C$}\sethlcolor{yellow} or \colorbox{pink}{$\dfrac{x}{A}\pm B=C$} where $\{A,B,C \in \mathbb{R}; A\neq 0\}$. We solve equations of this type by always moving the ``\hl{B}'' by adding or subtracting as appropriate, then multiplying or dividing to eliminate the ``\hl{A}''. Another way to think about solving these equations is to remove the ``excess baggage'' from the side with the variable, starting with the furthest baggage away from the variable. Both methods will result in the same steps being performed when solving two-step equations.

\subsection{How to ...}
\begin{itemize}
    \item Step 1. Move the number being added or subtracted from the term with the variable by performing inverse operations. \emph{You will always ADD or SUBTRACT in this step.}
    \item Step 2. Perform the inverse operation with the coefficient for the term with the variable. \emph{You will always MULTIPLY or DIVIDE in this step.}
\end{itemize}
\begin{example}{Solve Two-Step Equations Using Division}{}
    Solve $3x-5=13$ for $x$.
    \begin{enumerate}
        \item[] Add $5$ to both sides of the equation and simplify
        \begin{align*}
            3x-5&=13&\textcolor{red}{\textrm{   Original Equation}}\\
            3x-5+5&=13+5&\textcolor{red}{\textrm{   Addition Property of Equality (Add 5)}}\\
            3x+0&=18 &\textcolor{red}{\textrm{   Simplify (Combine Like Terms)}}\\
            3x&=18&\textcolor{red}{\textrm{   Identity Property of Addition}}\\            
        \end{align*}
        \item[] Divide both sides of the equation by $3$ and simplify
        \begin{align*}            
            \dfrac{3}{3}x&=\dfrac{18}{3}&\textcolor{red}{\textrm{   Division Property of Equality}}\\
            1\cdot x&=6&\textcolor{red}{\textrm{   Simplify (Combine Like Terms)}}\\
            x&=6&\textcolor{red}{\textrm{   Identity Property of Multiplication}}
        \end{align*}
        \item[] You may alternately show your work as follows:
        \begin{align*}        
            3x-5&=13\\
            +5& \;\;\;+5\\
            3x&=\;\;\;18\\
            ---&\;\;\;---\\
            3&\;\;\;\;\;\;3\\\\
            x=6
        \end{align*}        
    \end{enumerate}
\end{example}
\begin{example}{Solve Two-Step Equations Using Multiplication by a Fraction}{}
    Solve $\dfrac{4}{3}x+3=11$ for $x$.
    \begin{enumerate}
        \item[] Subtract $3$ to both sides of the equation and simplify
        \begin{align*}
            \dfrac{4}{3}x+3&=11&\textcolor{red}{\textrm{   Original Equation}}\\
            \dfrac{4}{3}x&=8&\textcolor{red}{\textrm{   Subtraction Property of Equality (Subtract 3)}}\\
        \end{align*}
        \item[] Multiply both sides of the equation by $\dfrac{3}{4}$ and simplify
        \begin{align*}
            \dfrac{3}{4}\cdot\dfrac{4}{3}x&=8\cdot\dfrac{3}{4}&\textcolor{red}{\textrm{   Multiplication Property of Equality}}\\
            x&=6&\textcolor{red}{\textrm{   Simplify}}
        \end{align*}      
    \end{enumerate}
\end{example}
\begin{example}{Solve Two-Step Equations Using Multiplication }{}
    Solve $\dfrac{x}{3}+5=11$ for $x$.
    \begin{enumerate}
        \item[] Subtract $5$ to both sides of the equation and simplify
        \begin{align*}
            \dfrac{x}{3}+5&=11&\textcolor{red}{\textrm{   Original Equation}}\\
            \dfrac{x}{3}&=6&\textcolor{red}{\textrm{   Subtraction Property of Equality (Subtract 5)}}\\
        \end{align*}
        \item[] Multiply both sides of the equation by 3 and simplify
        \begin{align*}
            3\cdot\dfrac{x}{3}&=3\cdot6&\textcolor{red}{\textrm{   Multiplication Property of Equality (Multiply by 3)}}\\
            x&=18&\textcolor{red}{\textrm{   Simplify}}
        \end{align*}      
    \end{enumerate}
\end{example}