\section{Verify Solutions of Equations}
We can verify whether the given number or expression is a solution to the equation by substituting this value (or expression) into the original equation to see if it makes a true statement when substituted into the equation.
\subsection{How to ... }
\begin{itemize}
    \item Step 1. Substitute the number in for the variable in the equation. 
    \item Step 2. Simplify the expressions on both sides of the equation. 
    \item Step 3. Determine whether the resulting equation is true (the left side is equal to the right side)
    \begin{itemize}
        \item If it is true, the number is a solution.
        \item If it is not true, the number is not a solution.
    \end{itemize} 
\end{itemize}

\begin{example}{Verify Solution of Linear Equation}{}
    Determine whether $-\dfrac{9}{5}$ is a solution to the equation $3x+4=-2x-5$.
	\begin{itemize}
		\item[] Write the original equation replacing the variable(s) we are verifying with parenthesis
              \begin{equation*}
                  3\left(\;\;\;\right)+4=-2\left(\;\;\;\right)-5
              \end{equation*}
 		\item[] Substitute the provided value into the parenthesis
              \begin{equation*}
                  3\left(-\dfrac{9}{5}\right)+4=-2\left(-\dfrac{9}{5}\right)-5
              \end{equation*}    
            \item[] Simplify each side following the order of operations
              \begin{align*}
                  \left(-\dfrac{27}{5}\right)+4&=\left(\dfrac{18}{5}\right)-5 &\textcolor{red}{\textrm{   Distribute into the parenthesis}}\\
                  -\dfrac{27}{5}+\dfrac{20}{5}&=\dfrac{18}{5}-\dfrac{25}{5} &\textcolor{red}{\textrm{   Rewrite with common denominators}}\\
                  -\dfrac{7}{5}&= -\dfrac{7}{5} &\textcolor{red}{\textrm{  Combine like terms }}
              \end{align*}
            \item[]Determine whether the resulting equation is true\\
            (\true)
	\end{itemize}
 $-\dfrac{9}{5}$ is a solution to the equation.
\end{example}
\newpage
We can extend this to systems of equations (we covered this in 8th grade math.
\begin{example}{Verify Solution of Linear System}{}
    Verify that the point $(3,2)$ is a solution to the linear system
    \begin{align*}
        y&=-\dfrac{2}{3}x+4\\
        y&=3x-7
    \end{align*}    
    \begin{itemize}
	\item[] Write the original equation replacing the variable(s) we are verifying with parenthesis
        \begin{align*}
            \left(\;\;\;\right)&=-\dfrac{2}{3}\left(\;\;\;\right)+4\\
            \left(\;\;\;\right)&=3\left(\;\;\;\right)-7
        \end{align*}    
        \item[] Substitute the provided values $\left(x=3, y=2\right)$ into the parenthesis
        \begin{align*}
            (2)&=-\dfrac{2}{3}(3)+4\\
            (2)&=3(3)-7
        \end{align*}   
        \item[] Simplify each side of each equation following the order of operations
        \begin{align*}
            2&=-2+4&\textcolor{red}{\textrm{   Distribute into the parenthesis}}\\
            2&=9-7&\textcolor{red}{\textrm{   Distribute into the parenthesis}}
        \end{align*}   
         \begin{align*}
            2&=2&\textcolor{red}{\textrm{   Combine like terms}}\\
            2&=2&\textcolor{red}{\textrm{   Combine like terms}}
        \end{align*}  
            \item[]Determine whether the resulting equation is true\\
            (\true)
    \end{itemize}
    The point $(3,2)$ is a solution to the system of equations
\end{example}

