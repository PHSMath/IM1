\section{Solving Literal Equations}
Literal equations are those that have more than one variable. Another name for a literal equation would be formula. We typically want to solve a literal equation for a specific variable if we want to do several problems that require us to find values of that variable. This is especially important in your science classes. For example, the distance an object travels is related to the time traveled as follows:
\begin{align*}
    d&=r\cdot t
\end{align*}
where $d$ is the distance traveled, $r$ is the rate of travel, and $t$ is the time traveled. Some times you will need to solve many problems to get the rate. Instead of solving each one individually, we can solve the literal equation (formula) for the variable of interest, then use substitution to find the answers to our problems. In this example, the literal equation would become:
\begin{align*}
    r&=\frac{d}{t}
\end{align*}
\section{How to ...}
\begin{itemize}
    \item Identify the variable of interest. 
    \item Move every term with that variable to one side of the equation with everything else on the other. This can be done in multiple methods following the steps from the previous sections.
\end{itemize}
\begin{example}{Solving Literal Equations - Basic Example}{}
    Solve $A(Bx-C)+D=E$ for $x$
    \begin{align*}
        A(Bx-C)+D&=E&\;\;\;\textcolor{red}{\textrm{Original Equation}}\\
        A(Bx-C)&=E-D&\;\;\;\textcolor{red}{\textrm{Subtraction Property of Equality (Subtract $D$ from both sides)}}\\
        Bx-C &=\dfrac{E-D}{A}&\;\;\;\textcolor{red}{\textrm{Division Property of Equality (Divide both sides by $A$})}\\
        Bx &= =\dfrac{E-D}{A}+C&\;\;\;\textcolor{red}{\textrm{Addition Property of Equality (Add $C$ to both sides)}}\\
        x &= \dfrac{\dfrac{E-D}{A}+C}{B}&\;\;\;\textcolor{red}{\textrm{Division Property of Equality (Divide both sides by $B$})}
    \end{align*}
    For the purposes of this course, this would be a sufficient answer. In future courses you will be taught how to simplify the complex fraction.
\end{example} 
\begin{note}{}{}
You can perform any LEGAL mathematical operation when solving equations. In later classes we will be adding 0 (add and subtract the same number), multiplying by 1 (multiply and divide by the same number), factoring (if literal equations have more than one variable term), and other methods to solve equations.
\end{note}