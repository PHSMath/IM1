\section{Solving One-Step (Simple) Equations}
We find \hl{\textbf{solutions}} to equations by using properties of real numbers, properties of equality, and \hl{\textbf{inverse operations}} to form \hl{\textbf{equivalent equations}}(equations that have the same solutions), until we get our variable of interest by itself. This will give us a new equation where our variable is equal to a number (or an expression). 

\begin{table}[H]
	\centering
	\caption{Properties of Equality.}
	\label{tab:Prop_of_Equal_Table}
	\begin{NiceTabular}{llll}[
			cell-space-limits=5pt, code-before=\rowcolors{1}{\tabcol!15}{\tabcol!10} \rowcolor{\tabcol!50}{1}
		]
		\toprule
		\RowStyle[bold=true]{}If $a=b$, then & $\{a,b,c \in \mathbb{R}; c\neq 0\}$\\
		\midrule
		$a+c=b+c$ & Add same number to both sides of equation \\
		$a-c=b-c$ & Subtract same number to both sides of equation \\
		$a\cdot c=b\cdot c$ & Multiply both sides of equation by same number \\
		$\dfrac{a}{c}=\dfrac{b}{c}$ & Divide both sides of equation by same number \\
		\midrule
	\end{NiceTabular}
\end{table}

One-step equations are those that only require a single mathematical operation in order to solve (well, actually more, but only one that needs to be shown). We are going to do the same thing to both sides of the equation in order to get the variable by itself.

\FloatBarrier

\subsection{How to...}

\begin{itemize}
    \item Step 1. Determine the number that needs to be moved.
    \item Step 2. Perform the inverse operation with that number to both sides of the equation
    \begin{itemize}
        \item Addition / Subtraction
        \item Multiplication / Division
    \end{itemize} 
\end{itemize}
\begin{example}{Solve One-Step Equation Using Subtraction}{}
    Solve $x+5=12$ for $x$.
    \begin{enumerate}
        \item[] Subtract $5$ from both sides of the equation and simplify
        \begin{align*}
            x+5&=12&\textcolor{red}{\textrm{   Original Equation}}\\
            x+5-5&=12-5&\textcolor{red}{\textrm{   Subtraction Property of Equality (Subtract 5)}}\\
            x+0&=7 &\textcolor{red}{\textrm{   Simplify (Combine Like Terms)}}\\
            x&=7&\textcolor{red}{\textrm{   Identity Property of Addition}}
        \end{align*}
        \item[] You may alternately show your work as follows:
        \begin{align*}        
            x+5&=12\\
            -5& \;\;\;-5\\
            x&=\;\;\;7
        \end{align*}        
    \end{enumerate}
\end{example}
Even though you do not need to show all of the work as in the first part in Example 1.3, you are expected to understand that you are actually performing these operations when you solve equations. We will not show all of these steps in the following examples as they were thoroughly shown and required during 8th grade math.
\newpage
\begin{example}{Solve One-Step Equation Using Division}{}
    Solve $4x=56$ for $x$. (There is an invisible $\cdot$ between the $4$ and the $x$.
    \begin{enumerate}
        \item[] Divide both sides by $4$ and simplify
        \begin{align*}        
            4x&=56\\
            \dfrac{4}{4}x&=\dfrac{56}{4}\\
            x&=14
        \end{align*}        
    \end{enumerate}
\end{example}
\begin{note}{Fractions as Answers}{}
    Unless otherwise directed, you may leave fractions as simplified improper fractions. Do not turn them into mixed numbers or decimals.
\end{note}
\begin{example}{Solve One-Step Equation Using Division - With Fraction}{}
    Solve $6x=92$ for $x$.
    \begin{enumerate}
        \item[] Divide both sides by $6$ and simplify
        \begin{align*}        
            6x&=92\\
            \dfrac{6}{6}x&=\dfrac{92}{6}\\
            x&=\dfrac{46}{3}
        \end{align*}        
    \end{enumerate}
\end{example}
If the variable we are solving for has a fractional coefficient, we need to multiply both sides of the equation by the reciprocal of the coefficient.
\begin{example}{Solve One-Step Equation Using Multiplication by a Fraction}{}
    Solve $\dfrac{2}{3}x=20$ for $x$. (Can also be written as $\dfrac{2x}{3}=20$)
    \begin{enumerate}
        \item[] Multiply both sides by $\dfrac{3}{2}$ and simplify
        \begin{align*}        
            \dfrac{2}{3}x&=20\\
            \dfrac{3}{2}\cdot\dfrac{2}{3}x&=20\cdot\dfrac{3}{2}\\
            x&=\dfrac{60}{2}\\
            x&=30
        \end{align*}        
    \end{enumerate}    
\end{example}
