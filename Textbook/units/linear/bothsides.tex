\section{Solving Equations with Multiple Variable Terms}
We have been focusing on equations that only had a single variable term. We can have equations that have multiple variable terms. Either multiple variables on one side of the equation, variables on both sides of the equation, or both. In order to solve equations of this type we will be ``moving'' every term with a variable to one side of the equation and every term without a variable to the other side of the equation using inverse operations. Some of these problems may also include distribution.

If you understand the order of operations and inverse operations, these problems are only slightly more complicated than those we have previously completed.
\subsection{How to ...}
\begin{itemize}
    \item \textbf{Distribute} into any parenthesis to remove the parenthesis. Remember that $-(x-5)$ has an invisible $1$ next to the parenthesis so the result of the distribution would be $-x+5$.
    \item \textbf{Combine} like terms on the SAME side of the equation. Like terms are those that have the exact same variable(s) to the exact same power(s). The following are like terms: $x, 5x, -17x$. The following are NOT like terms: $x, 3x^2, -2xy$.
    \item \textbf{Move} the variable terms to one side of the equation and the terms without variables to the other side of the equation using \textbf{Addition} or \textbf{Subtraction}. We normally move the variable to the side that originally had the greater quantity of the variable.
    \item \textbf{Isolate} the variable to get a coefficient of $1$ using \textbf{Multiplication} or \textbf{Division}. This is the exact same as solving a one-step equation.
\end{itemize}
\begin{example}{Variables on Both Sides - No Distribution}{}
    Solve $x+2=2x-7$ for x.
    \begin{itemize}
        \item Distribute - Nothing to Perform
        \item Combine Like Terms - Nothing to Perform
        \item Move (We will be moving variable to right, constants to left)
            \begin{align*}
                x+2&=2x-7&\;\;\;\textcolor{red}{\textrm{Original Equation}}\\
                2&=x-7&\;\;\;\textcolor{red}{\textrm{Subtraction Property of Equality (Subtract $x$ from both sides)}}\\
                9&=x&\;\;\;\textcolor{red}{\textrm{Addition Property of Equality (Add 7 to both sides)}}
            \end{align*}
    \end{itemize}
    You can perform both steps at the same time.
\end{example}
\begin{example}{Variables on Both Sides - with Distribution}{}
    Solve $3(2x-1)+4=2(4x=3)-4$ for x.
    \begin{itemize}
        \item Distribute
        \begin{align*}
            3(2x-1)+4&=2(4x-3)-4&\;\;\;\textcolor{red}{\textrm{Original Equation}}\\
            6x-3+4&=8x-6-4&\;\;\;\textcolor{red}{\textrm{Distribution Property}}\\
        \end{align*}
        \item[] Combine Like Terms
        \begin{align*}    
            6x+1&=8x-10&\;\;\;\textcolor{red}{\textrm{Simplify}}\\
        \end{align*}
        \item[] Move (We will be moving variable to right, constants to left)
        \begin{align*}    
            11&=2x&\;\;\;\textcolor{red}{\textrm{Add 10 to and Subtract 6x from both sides of equation}}\\
        \end{align*}
        \item[] Isolate
        \begin{align*}    
            \dfrac{11}{2}&=x&\;\;\;\textcolor{red}{\textrm{Division Property of Equality (Divide both sides by 2)}}\\
        \end{align*}        
    \end{itemize}
\end{example}